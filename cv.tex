\documentclass[12pt,a4paper]{moderncv}
\moderncvtheme[purple]{classic}                
\usepackage[utf8]{inputenc}
\usepackage[inline]{enumitem}
\usepackage[top=1.0cm, bottom=1.0cm, left=1.6cm, right=1.6cm]{geometry}
\setlength{\hintscolumnwidth}{2.7cm}

\firstname{Maxence}
\familyname{Brochier}
\title{Administrateur système Junior}              
\address{2720 Route de l'Eculaz}{74930 Reignier-Esery}    
\email{maxence@brochier.xyz}                      
\homepage{https://brochier.xyz}
\mobile{06 10 46 31 77} 
\photo{profil.jpeg}

\begin{document}
	\maketitle
	
	\vspace*{-2.5\baselineskip}
	
	\section{Expériences}
	\cventry{Juil 2021\\à Aujourd'hui}{Administrateur réseau}{Nextvision}{Annemasse}{France}{
		\begin{itemize}%
			\item Gestion du réseau de l'entreprise \texttt{/} des clients
			\item Dépanage à par téléphone  \texttt{/} sur site  \texttt{/} remote desktop
			\item Paramétrage et configuration des différents équipements réseaux et audio-visuels  \texttt{/} des solutions de contrôle unifiés
			\item \textbf{Développement} : \textit{scripts bash, web html\texttt{/}css\texttt{/}php\texttt{/}mysql} : Laravel
			\begin{itemize}%
				\item Système de fiche d'intervention interne, utilisateurs, droits, permissions, exportation PDF, validation, édition, historique et logs.\newline
			\end{itemize}
	\end{itemize}}
	
	
	\cventry{Sept 2020\\à Juil 2021}{Technicien Informatique}{2LINFO}{Annemasse}{France}{
		\begin{itemize}
			\item Reprise du code de l'API en \textit{Golang}, amélioration et implémentation de nouvelles fonctionnalités.
			\item Migration des factures existante dans le compte Stripe de la société dans le système interne pour automatiser la facturation et la mise à jour de ses dernières (\textit{Stripe API})
	\end{itemize}}

	
	\section{Formations}
	\cventry{2019 -- 2021}{DUT informatique}{Université Polytechnique Haut-de-France / Maubeuge (59)}{}{}{Formation orienté développement informatique\texttt{/}C, Bash, Python, Java, HTML, CSS, PHP, Javascript}
	\cventry{2017 -- 2019}{Baccalauréat STI2D SIN}{Lycée Louis-Lachenal / Annecy (74)}{}{}{Spécialité systèmes d'informations numériques.\texttt{/}Debian et base de la programmation JAVA en plus}
	
	\section{Certifications}
	\cventry{2022}{Certification LPI Essentials}{Linux Professionnal Institute}{France}{\href{}{}}{
		Certification \textit{LPI Essentials} créé par \textit{Linux Professionnal Institute}. 
		Les compétences, une fois validées, attestent de la connaissance basique des systèmes linux.\newline
	}
	\cventry{2022}{Certification CISCO CCNA 1 \& 2}{CISCO}{France}{}{
		Certification \textit{CISCO CCNA}.
	}
	
	\section{Compétences en informatique}
	\cvdoubleitem{\underline{OS}}{Ubuntu, Debian, Windows}{\underline{Virtualisation}}{Proxmox, OpenVZ, LXC}
	\cvitem{\underline{Programmation}}{HTML, CSS, Javascript, PHP, SQL, Bash, C, Golang, Python}
	\cvitem{\underline{Framework}}{Laravel, Tailwind, Angular, ArduinoGo}
	\cvitem{\underline{Admin. services}}{Apache, Nginx, MySQL, Postfix, Systemctl}
	\cvitem{\underline{Admin. fichiers}}{LVM, BRTFS (\textit{Synology}), RAID, SHR (\textit{Synology})}
	\cvlanguage{\underline{Anglais}}{Toeic 761 \texttt{/} 990}{}
	
	\section{Centres d'intérêt}
	\cvitem{}{Sport en montagne, sports aquatique, Open-Source, Voyages}
\end{document}
